\begin{table}[htbp]
  \centering
  \caption{Taxonomic placement of the species currently included in the sections \textit{Casparya} according to \cite{Klotzsch1855}}
    \begin{tabular}{rp{7.715em}p{21.355em}}
    \toprule
    \multicolumn{1}{l}{Family} & \multicolumn{1}{l}{Genus} & \multicolumn{1}{l}{Species included} \\
    \midrule
    \multicolumn{1}{r}{\multirow{Begoniaceae}} & \multirow{\textit{Stiradotheca Kl.}} & \multirow{\textit{Stiradotheca magnifica, S. ferruginea, S. trachyptera}} \\
          & \multicolumn{1}{r}{} & \multicolumn{1}{r}{} \\
\cmidrule{2-3}          & \textit{Casparya Kl. } & \textit{Casparya hirta, C. coccinea (= B. urticae), C. columnaris (= B. urticae), C. elegans, (=B. foliosa (sect. Lepsia))} \\
\cmidrule{2-3}          & \textit{Isopteris} Kl. & \textit{Isopteris umbellate, I. longirostris} \\
\cmidrule{2-3}          & \textit{Sassea} Kl.  & \textit{Sassea columnaris, S. glabra, S. urticae} \\
    \bottomrule
    \end{tabular}%
  \label{tab:addlabel}%
\end{table}%