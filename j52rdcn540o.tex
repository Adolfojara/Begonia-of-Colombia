\begin{table}[htbp]
  \centering
  \caption{Taxonomic placement of the species currently included in the sections \textit{Casparya} according to De Candolle. In gray species not currently in the section Casparya}
    \begin{tabular}{rp{7.715em}p{21.355em}}
    \toprule
    \multicolumn{1}{p{5.355em}}{Genus} & Section & species included \\
    \midrule
    \multicolumn{1}{r}{\textit{Casparya}} & \textit{Stiradotheca Kl.} & \textit{Casparya ferruginea} [=\textit{Stiradotheca magnifica}], \textit{C. fuchsiaeflora} \\
\cmidrule{2-3}          & \textit{Isopteris Kl.} & \textit{C. umbellate, C. antioquensis} \\
\cmidrule{2-3}          & \textit{Sassea Kl.} & \textit{C. urticae, C. columnaris, C. montana, C. cordifolia, C. brevipetala, C. trachiptera} \\
\cmidrule{2-3}          & \textit{Aetheropterix A.DC.} & \textit{C. trispathulata} \\
\cmidrule{2-3}          & \textit{Andiphila A.DC.} & \textit{C. trianei, C. grawiifolia, C. longirostris} \\
\cmidrule{2-3}          & \cellcolor[rgb]{ .906,  .902,  .902}\textit{Sphenanthera} & \cellcolor[rgb]{ .906,  .902,  .902}\textit{C. robusta, C. multangular, C. erosa, C. teysmanniana} \\
\cmidrule{2-3}          & \cellcolor[rgb]{ .906,  .902,  .902}\textit{Holoclinum} & \cellcolor[rgb]{ .906,  .902,  .902}\textit{C. trisulcata} \\
\cmidrule{2-3}          & \cellcolor[rgb]{ .906,  .902,  .902}\textit{Polyschima} & \cellcolor[rgb]{ .906,  .902,  .902}\textit{C. crassicaulis} \\
    \bottomrule
    \end{tabular}%
  \label{tab:candolle}%
\end{table}%